The following figure shows a sketch of the game of life simulation implemented with Gray\+Bat support. The Game of Life domain (20 x 20 cells) is decomposed very fine grain, such that, every cell is represented by a vertex in a two-\/dimensional grid and neighboring vertices are connected by edges (\hyperlink{structgraybat_1_1pattern_1_1GridDiagonal}{graybat\+::pattern\+::\+Grid\+Diagonal}). The graph is partioned into four partitions and mapped to four peers (graybat\+::pattern\+::\+Graph\+Partition). Therefore, each peer is responsible for the communication of 100 vertices with its neighbors. Finally, each two peers are mapped to a quad core processor on a dual socket system (\hyperlink{structgraybat_1_1communicationPolicy_1_1BMPI}{graybat\+::communication\+Policy\+::\+B\+M\+P\+I}). This is one possible sequence of mappings for a Game of Life simulation within the Gray\+Bat framework. On each step changes are possible to adapt the simulation to other architectures, networks or algorithms.



\subsection*{See Also}


\begin{DoxyItemize}
\item \hyperlink{gol_8cpp-example}{Go\+L Sources}
\item \hyperlink{cage}{Communication and Graph Environment}
\item \hyperlink{communicationPolicy}{Communication Policy}
\item \hyperlink{graphPolicy}{Graph Policy}
\item \hyperlink{communicationPattern}{Communication Pattern}
\item \hyperlink{mapping}{Vertex Mapping} 
\end{DoxyItemize}